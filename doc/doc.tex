\documentclass[12pt,a4paper]{article}
\usepackage[utf8]{inputenc}
\usepackage{amsmath}
\usepackage{amsfonts}
\usepackage{amssymb}
\usepackage{graphicx}
%\usepackage[left=2cm,right=2cm,top=2cm,bottom=2cm]{geometry}
\author{José Antonio García-Hernández}
\title{Documentation}
\usepackage{tabularx}
\begin{document}
\maketitle

\section{Main}
This is the main program. It calls the thermalization subroutine.
\section{Parameters}
This is a module.
\section{Arrays}
\section{Datatypes} 
\section{Dynamics}
\subsection{Subroutine: hot\_start(U)}
 This subroutine creates a random SU(2) field configuration $U$ on the lattice.
 \subsection{Subroutine: cold\_start(U)}
 This subroutine creates a uniform SU(2) field configuration $U$ on the lattice equal to the 2$\times$2 identity matrix.
 \subsection{Function: SU2\_ran()}
 This function does not require an argument. It creates a random SU(2) element.
 \subsection{Function: small\_SU2\_ran()}
 This function does not require an argument. It creates a random SU(2) element near the identity. 
  \subsection{Function: sgn(x)}
 This function returns 1 if the argument $x$ is positive, -1 if negative and 0 otherwise. 
  \subsection{Function: SU2\_matrix(a,b)}
 This function creates a SU(2) matrix given two complex numbers $a$ and $b$.
  \subsection{Function: plaquette(U,x,mu,nu)}
 This function takes an SU(2) type element $U$ a four-element integer array $x$ and two integers $\mu$ and $\nu$ and returns a SU(2) type element.
   \subsection{Function: plaquette\_value(U)}
 This function takes an SU(2) type element $U$ a four-element and return a double precision real. This function computes the sum of the plaquettes of the whole lattice and divide it by the volume by $N=2$, and by the number of planes in $d = 4$ dimensions.
    \subsection{Function: staples(U,x,mu)}
 This function takes a 5-dimensional array of SU(2) type element $U$ a 4-element integer array $x$ and an integer $\mu$ and returns a SU2 type element. This function computes the sum of staples in $d =4$ dimensions.
     \subsection{Function: DS(U,Up,x,mu,beta)}
 This function takes a rank 5 array of SU(2) type element $U$ another SU2 type Up, a 4-element integer array $x$, an integer $\mu$, a double precision real beta and returns a double precision number. This function computes the difference of the action after a single link update.
	\subsection{Subroutine: thermalization(U, beta)}
	This subroutine updates the lattice by $N_{thermalization}$ sweeps to thermalize.
	\subsection{Subroutine: measurements(U, beta,P)}
	This function takes measurements of the plaquette value every $N_{skip}$ sweeps $N_measurements$ times.	 
	
	\subsection{Subroutine: sweeps(U, beta)}
	This subroutine goes through all the lattice and all the directions and applies the local update algorithm (metropolis or heatbath).	 
	
	\subsection{Subroutine: metropolis(U,x,mu,beta)}
	This subrotuine implements the metropolis algorithm.	 
 
 \subsection{Subroutine: heatbath(U,x,mu,beta)}
 \begin{tabular}{rll}
 subroutine heatbath(& type(SU2), dimension(:,:,:,:,:) & U, \\ 

  & integer, dimension(4) & x, \\ 

  & integer & mu, \\ 

  & real(dp) & beta \\ 
  ) & &
 \end{tabular} 
 
 Purpose:\\
 This subroutine implements the heatbath algorithm.
 
 Parameters:
 	
 \begin{tabularx}{\textwidth}{llX}
  \textbf{[inout]} & U & U is a derived data type SU2 array, dimension(:,:,:,:,:). It updates the link variable U(mu,x(1),x(2),x(3),x(4)). \\
  \textbf{[in]} & x & x is an INTEGER array, dimension (4). It is the array of the points in the lattice. \\ 
  \textbf{[in]} & mu & mu is an INTEGER. Is the direction of the link.  \\ 
  \textbf{[in]} & beta & beta is a REAL of kind (dp). It is related to the gauge coupling.
  \end{tabularx}  
  
  
\end{document}
