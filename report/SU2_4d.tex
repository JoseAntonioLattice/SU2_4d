\documentclass[12pt,a4paper]{article}
\usepackage[utf8]{inputenc}
\usepackage{amsmath}
\usepackage{amsfonts}
\usepackage{amssymb}
\usepackage{graphicx}

%matrix one
\usepackage{dsfont}

%
\usepackage{tikz}
\usetikzlibrary{math} %needed tikz library
\usetikzlibrary{arrows,decorations.markings}

\author{José Antonio García Hernández}
\title{4-dimensional SU(2) lattice gauge theory}
\begin{document}
\maketitle

\section{Introduction}
An SU(2) link variable is defined at a lattice point $x$ in the $\mu$ direction as $U _{\mu}(x)\in \text{SU}(2)$. An element of the group SU(2) is defined as follows

\begin{equation}
	\label{eq:SU2_element}
	U = \begin{pmatrix}
		a & b \\
		-b^* & a^*
	\end{pmatrix}
\end{equation}
where $a, \ b \in \mathbb{C}$ and $|a|^2 + |b|^2 = 1$.

\subsection{Action}
In the continuum, the action reads
\begin{equation}
	\label{eq:continuum_action}
	S[A] = \frac{1}{2g^2} \int d^4x \, \text{Tr} \left[ F_{\mu\nu}(x)F_{\mu\nu}(x)\right],
\end{equation}
where $F_{\mu\nu}(x)$ is the field strength tensor.

We defined the plaquette variable on the lattice as the ordered product of link variables on the lattice as follows
\begin{equation}
	\label{eq:plaquette}
	U_{\mu\nu}(x) = U_{\mu}(x)U_{\nu}(x+\hat{\mu})U_{\mu}^{\dagger}(x+\hat{\nu})U_{\nu}^{\dagger}(x) 
\end{equation}


On the lattice, the action takes the form
\begin{equation}
	\label{eq:wilson_action}
	S[U] = \frac{\beta}{N}\sum_x \sum_{\mu < \nu} \text{Re}\ \text{Tr} \left[\mathds{1} - U_{\mu\nu}(x) \right],
\end{equation}
where $\beta = \frac{2N}{g^2}$.

An important quantity is the sum of the plaquettes
\begin{equation}
	\label{eq:Sp}
	S_{\text{P}} = \frac{1}{N} \sum_x\sum_{\mu < \nu} \text{Re}\ \text{Tr} [U_{\mu\nu}(x)],
\end{equation}
and we define the plaquette value as
\begin{equation}
	\label{eq:Ep}
	P =\frac{ \langle S_{\text{P}} \rangle}{D V},
\end{equation}
where $V=L^d$ is the volume of the lattice and $D = \frac{d(d-1)}{2}$ is the number of planes of rotation.

The sum of the staples of a link in the direction $\mu$ is
\begin{equation}
	\label{eq:staples}
	\Sigma_{\mu}(x) = \sum_{\nu \neq \mu} \left[ U_{\nu}(x)U_{\mu}(x+\hat{\nu})U_{\nu}^{\dagger}(x+\hat{\mu}) + U_{\nu}^{\dagger}(x-\hat{\nu})U_{\mu}(x-\hat{\nu})U_{\nu}(x+\hat{\mu}-\hat{\nu})\right].
\end{equation}


The change in the action  by a local update when changing $U_{\mu}(x) \to U'_{\mu}(x)$ is
\begin{equation}
	\label{eq:DS}
	\Delta S = -\frac{\beta}{N} \text{Re } \text{Tr} \left[ \left( U'_{\mu}(x) - U_{\mu}(x) \right)\Sigma_{\mu}^{\dagger}\right].
\end{equation}

\subsection{Field strength tensor}

We can define the field strengh tensor on the lattice with either the \emph{Plaquette} or \emph{Clover} definition. A clover is defined as follows
\begin{eqnarray}
    Q_{\mu\nu}(x) & = & U_{\mu}(x)U_{\nu}(x+\hat{\mu})U_{\mu}^{-1}(x+\hat{\nu})U_{\nu}^{-1}(x)  +  \nonumber\\
    & & U_{\nu}(x)U_{\mu}^{-1}(x-\hat{\mu}+\hat{\nu})U_{\nu}^{-1}(x-\hat{\mu})U_{\mu}(x-\hat{\mu})  + \nonumber\\ 
    & & U_{\mu}^{-1}(x-\hat{\mu})U_{\nu}^{-1}(x-\hat{\mu}-\hat{\nu})U_{\mu}(x-\hat{\mu}-\hat{\nu})U_{\nu}(x-\hat{\nu})+\nonumber\\
    &  & U_{\nu}^{-1}(x-\hat{\nu}) U_{\mu}(x-\hat{\nu})U_{\nu}(x+\hat{\mu}-\hat{\nu})U_{\mu}^{-1}(x).
\end{eqnarray}

\tikzmath{\eps= 0.05;}
\begin{figure}
\begin{center}
    \begin{tikzpicture}[decoration={markings,mark= at position 0.15 with {\arrow[scale=2,>=latex]{>}},markings,mark= at position 0.4 with {\arrow[scale=2,>=latex]{>}},mark= at position 0.65 with {\arrow[scale=2,>=latex]{>}},mark= at position 0.9 with {\arrow[scale=2,>=latex]{>}}},scale=2.3]
        \draw[postaction={decorate}]  (\eps,\eps)--(1-\eps,\eps)--(1-\eps,1-\eps)--(\eps,1-\eps)--(\eps,\eps);
        \draw[postaction={decorate}] (-\eps,\eps)--(-\eps,1-\eps)--(-1+\eps,1-\eps)--(-1+\eps,\eps)--(-\eps,\eps);
        \draw[postaction={decorate}] (-\eps,-\eps)--(-1+\eps,-\eps)--(-1+\eps,-1+\eps)--(-\eps,-1+\eps)--(-\eps,-\eps);
        \draw[postaction={decorate}] (\eps,-\eps)--(\eps,-1+\eps)--(1-\eps,-1+\eps)--(1-\eps,-\eps)--(\eps,-\eps); 
        
        %\fill[red] (0,0) circle(\eps);
        \fill (1,0) circle(\eps);
        \fill (1,1) circle(\eps);
        \fill (0,1) circle(\eps);
        \fill (-1,0) circle(\eps);
        \fill (0,-1) circle(\eps);
        \fill (-1,-1) circle(\eps);
        \fill (-1,1) circle(\eps);
        \fill (1,-1) circle(\eps);

        \draw (0,0) node[]{$x$};
        
        \draw (0,-1.3) node[below]{$Q_{\mu\nu}(x)$};
    \end{tikzpicture}
\end{center}    
\caption{Clover.}
\end{figure}

Field strength tensor with the plaquette definition
\begin{equation}
    F^{\text{plaquette}}_{\mu\nu}(x) = i\left[\mathds{1} - U_{\mu\nu}(x)\right].
\end{equation}

Field strength tensor with the clover definition
\begin{equation}
    F^{\text{clover}}_{\mu\nu}(x) = \frac{1}{8i}\left[ Q_{\mu\nu}(x) -  Q_{\nu\mu}(x)\right].
\end{equation}

\subsection{Energy density}
A possibility of defining the energy density on the lattice is by using the plaquette
\begin{equation}
    E = \frac{1}{3V}\sum_{x}\sum_{\mu<\nu} \text{Re}\,\text{Tr} \left[\mathds{1} - U_{\mu\nu}(x) \right],
\end{equation}
or by using the more symmetric definition with the clover
\begin{equation}
    E = \frac{1}{2V}\sum_x \text{Re}\,\text{Tr}\left[F^{\text{clover}}_{\mu\nu}F^{\text{clover}}_{\nu\mu} \right].
\end{equation}
\subsection{Topological charge}
The topological charge $Q[U]$ of a configuration $[U]$ is defined as the sum over the lattice of the topological charge density $q(x)$ 
\begin{equation}
    q(x) = -\frac{1}{32\pi^2}\varepsilon_{\mu\nu\rho\sigma} \text{Tr} \left\{ F_{\mu\nu}(x)F_{\rho\sigma}(x)\right\}.
\end{equation}
Again, we can define the topological charge with the clover or plaquette definitions of $F_{\mu\nu}(x)$. Using the symmetries of the clover definition we can reduce the number of sums to be computed. First, we note that only 24 elements of the Levi-Civita symbol are non-zero. We also note that
\begin{eqnarray}
    \varepsilon_{\mu\nu\rho\sigma} \text{Tr} \left\{ F^{\text{clover}}_{\mu\nu}(x)F^{\text{clover}}_{\rho\sigma}(x)\right\} & = & \varepsilon_{\nu\mu\rho\sigma} \text{Tr} \left\{ F^{\text{clover}}_{\nu\mu}(x)F^{\text{clover}}_{\rho\sigma}(x)\right\} \\
    \varepsilon_{\mu\nu\rho\sigma} \text{Tr} \left\{ F^{\text{clover}}_{\mu\nu}(x)F^{\text{clover}}_{\rho\sigma}(x)\right\} & = & \varepsilon_{\mu\nu\sigma\rho} \text{Tr} \left\{ F^{\text{clover}}_{\mu\nu}(x)F^{\text{clover}}_{\sigma\rho}(x)\right\} \\
    \varepsilon_{\mu\nu\rho\sigma} \text{Tr} \left\{ F^{\text{clover}}_{\mu\nu}(x)F^{\text{clover}}_{\rho\sigma}(x)\right\} & = & \varepsilon_{\rho\sigma\mu\nu} \text{Tr} \left\{F^{\text{clover}}_{\rho\sigma} (x)F^{\text{clover}}_{\mu\nu}(x)\right\}
\end{eqnarray} 
This reduces the number of effective sums by a factor of 8. From the original 24 sums we only need to perform 3.

Using $Q$ instead of $F$ we arrive at
\begin{equation}
    \text{Tr}\{F^{\text{clover}}_{\mu\nu}(x)F^{\text{clover}}_{\rho\sigma}(x)\} =\frac{1}{2^5} \text{Tr}\left\{Q_{\mu\nu}(x)[Q_{\rho\sigma}(x)-Q_{\rho\sigma}^{\dagger}(x)]\right\}
\end{equation}
We choose the 3 independent permutations of $\varepsilon_{\mu\nu\rho\sigma}$ to be in the indices $[1,2,3,4]$, $[1,3,2,4]$ and $[1,4,2,3]$. Therefore,
\begin{equation}
    q^{\text{clover}}(x) = -\frac{1}{128 \pi^2} \sum_{[\mu\nu\rho\sigma] \in\{ [1234], [1324], [1423]\}}
              \varepsilon_{\mu\nu\rho\sigma}\text{Tr}(Q_{\mu\nu}[Q_{\rho\sigma}-Q_{\rho\sigma}^{\dagger}]) 
\end{equation}

\section{Local update algorithms}

We implemented three local update algorithms, namely, Metropolis, Glauber and Heathbath.

\subsection{Metropolis}
\begin{enumerate}
	\item Given some gauge field configuration, we go through all the lattice points in a lexicographic way. At each point $x$ in the $\mu$ direction we generate a random SU(2) matrix $U'_{\mu}(x)$.
	
	\item We compute the sum of the staples and compute the change of the action according to eq.\ \eqref{eq:DS}. We generate a uniform random number $r\in [0,1)$ and accept the change if $r \leq p$, where $p$ is
	\begin{equation}
		p = \min (1, \exp(-\Delta S)).
\end{equation}	 
\end{enumerate}

\subsection{Glauber}
\begin{enumerate}
	\item Given some gauge field configuration, we go through all the lattice points in a lexicographic way. At each point $x$ in the $\mu$ direction we generate a random SU(2) matrix $U'_{\mu}(x)$.
	
	\item We compute the sum of the staples and compute the change of the action according to eq.\ \eqref{eq:DS}. We generate a uniform random number $r\in [0,1)$ and accept the change if $r \leq p$, where $p$ is
	\begin{equation}
		p = \frac{1}{e^{\Delta S} + 1}.
\end{equation}	 
\end{enumerate}

\subsection{Heatbath}\label{sec:heatbath}
In the heatbath algorithm we need to generate an update $X\in\text{SU}(2)$
	 \begin{equation}
	 	X = \begin{pmatrix}
	 		 x_0 + ix_1 & x_2 + ix_3 \\
	 		-x_2 + ix_3 & x_0 - ix_3
	 	\end{pmatrix}
	 \end{equation}
	 where $x_0, x_1, x_2,x_3 \in \mathbb{R}$
 following the distribution
\begin{equation}
	dP(X) = \frac{1}{2\pi^2} d\cos\theta\, d\varphi\, dx_0\, \sqrt{1-x_0^2}e^{\alpha\beta x_0}
\end{equation}
\begin{enumerate}
\item Given some field configuration at a point $x$ and direction $\mu$, we compute the sum of staples $\Sigma_{\mu}(x)$ in eq.\ \eqref{eq:staples}, $\alpha = \sqrt{\det \Sigma_{\mu}(x)}$, and define $V = \Sigma_{\mu}(x)/\alpha$
	\item We generate $X$ as follows:
	\begin{enumerate}
	\item	We generate three random numbers $u_i$, $i = 1,\ 2,\ 3$ in the interval [0,1) uniformly distributed and take $r_i = 1 - u_i$. Then
		 \begin{equation}
		 	\lambda^2 = -\frac{1}{2\alpha\beta}\left(\log(r_1) + \cos^2(2\pi r_2)\log(r_3)\right).
		 \end{equation}
	\item We accept the value of $\lambda$ which obey
		\begin{equation}
			s^2 \leq 1- \lambda^2,
		\end{equation}
		where $s$ is a uniformly distributed random number in the interval [0,1).
	\item Repeat the previous two steps until a value of $\lambda$ is accepted. The accepted value gives $x_0 = 1 - 2\lambda^2$.
	\item We take the norm of the vector $\vec{x} = (x_1,x_2,x_3)$ as $|\vec{x}| = \sqrt{1 - x_0^2}$.
	\item We take two uniformly distributed random numbers $t \in [0,1)$ and $\varphi\in [0,2\pi)$ 
	\begin{eqnarray}
		x_1 & = & 1 - 2 t \\
		x_2 & = & \sqrt{1 - x_1^2} \cos\varphi \\
		x_3 & = & \sqrt{1 - x_1^2} \sin\varphi	
	\end{eqnarray}		
	  
	\end{enumerate}
	\item Once $X$ is generated, the new link variable is $U = X V$.
\end{enumerate}


\section{Smoothing procedures}

In order to compute the topological charge, we first need to smooth the configuration. This can be achived by various methods. We describe a few.

\subsection{Gradient Flow}
\begin{equation}
    \frac{d}{dt}V_{\mu}(x,t) = Z_{\mu}(x,t)V_{\mu}(x,t) , \ \ \  \left. {V_{\mu}(x,t)}\right|_{t = 0} = U_{\mu}(x)
\end{equation}

\begin{equation}
         Z_{\mu}(x) = -\{U_{\mu}(x)\Sigma_{\mu}^{\dagger}(x)\}_{\text{TA}}
\end{equation}

\begin{equation}
    W_{\text{TA}} = \frac{W - W^{\dagger}}{2} - \frac{\text{Tr} (W - W^{\dagger})}{2N}
\end{equation}

\begin{equation}
    V_{\mu}(x,t+\varepsilon) = \exp\{\varepsilon Z_{\mu}(x,t)\}V_{\mu}(x,t).
\end{equation}
%
%\begin{equation}
%    E(x,t) = \frac{1}{4} \hat{F}_{\mu\nu}^a(x,t) \hat{F}_{\mu\nu}^a(x,t)
%\end{equation}    


%\begin{equation}
%    \mathcal{E}(t) = t^2 \langle E(x,t)\rangle
%\end{equation}

\subsection{Cooling}
Each link is updated by
\begin{equation}
    U_{\mu}'(x) = \frac{\Sigma_{\mu}(x)}{\sqrt{\det \Sigma_{\mu}(x)}}.
\end{equation}
\subsection{APE smearing}
Each link is updated by 
\begin{equation}
    U_{\mu}'(x) = (1-\alpha)U_{\mu}(x)+ \frac{\alpha}{6}\Sigma_{\mu}(x).
\end{equation}
with $0<\alpha\leq 0.75$. Only in SU(2) this update stays in the gauge group. Otherwise we need to project it back to the gauge group.% We chose $\alpha = 0.55$.

\section{Results}
We present simulation results with 20 configurations of the gauge field at $\beta = 1.95$ and $V = 10^4$. After 100 sweeps of thermalization, we procceed to smooth the configurations applying \emph{cooling} or \emph{Gradient Flow}. Figure \ref{fig:cooling} shows the topological charge vs. the sweeps of cooling over the lattice. We see plateaus of the topological charge that are nearly an of integer value. They are almost an intger when $|Q| \leq 2$. In figure \ref{fig:gradient_flow} we also see plateaus of integer values, they appear when $t>15$. Again, the plateaus are almost an integer when $|Q|\leq 2$.
\begin{figure}

    \includegraphics[scale=1]{../data/cooling.pdf}
    \caption{Topological charge $Q$ vs. the sweeps of cooling.}
    \label{fig:cooling}
\end{figure}

\begin{figure}
    \includegraphics[scale=1]{../data/gradient_flow.pdf}
    \caption{Topological charge $Q$ using gradient flow to smooth the configurations vs. the \emph{Langevin time} $t$. Each colored line represents the evolution of a single configuration with a volume $V = 10^{4}$.}
    \label{fig:gradient_flow}
\end{figure}

\begin{center}
\begin{figure}
    \includegraphics[scale=0.5]{../data/energyden.pdf}
    \caption{$\langle E \rangle t^2$ vs. the Langevin time $t$ using the definitions with the plaquette and the clover.}
\end{figure}
\end{center}
\end{document}
